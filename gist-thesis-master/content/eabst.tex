% -*- coding: utf-8 -*-

This thesis describes GMES, a free Python package for solving Maxwell's equations using the finite-difference time-domain (FDTD) method. The design of GMES follows the object-oriented programming (OOP) approach and adopts a unique design strategy where the voxels in the computational domain are grouped and then updated according to its material type. This piecewise updating scheme ensures that GMES can adopt OOP without losing its simple structure and time-stepping speed. The users can easily add various material types, sources, and boundary conditions into their code using Python programming language. The key design features, along with the supported material types, excitation sources, boundary conditions and parallel calculations employed in GMES are also described in detail.

Also, we highlights the implementation of Drude-critical point model for describing dispersive media into finite difference time domain algorithm using piecewise-linear recursive-convolution and auxiliary differential equation method. The advantages, accuracy and stability of both implementations are analyzed in detail. Both implementations are applied in studying the transmittance and reflectance of thin metal films, and excellent agreement is observed between the analytical and numerical results.
