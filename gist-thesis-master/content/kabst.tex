% -*- coding: utf-8 -*-

본 논문은 맥스웰 방정식의 수치적 해를 유한차분시간영역(finite-difference time-domain, FDTD) 방법으로 구하는 파이썬 패키지인 GMES에 관해 다룬다. GMES는 매질의 종류에 따라 복셀(voxel)을 분류한 후, 이에 따라 전자기파를 업데이트하는 독특한 방식을 반영한 객체지향적프로그래밍(object-oriented programming, OOP) 방법으로 설계되었다. 구분별 갱신 방식(piecewise updating scheme)으로 명명한 이 방식은 FDTD 특유의 간단한 기본 설계와 비교적 빠른 계산 속도를 유지하면서도 OOP를 적용한 매우 유연한 설계가 가능하도록 해주었다. 현재까지 개발된 GMES는 사용자가 파이썬 언어로 자유롭게 매질, 입력 광원, 경계 조건 등을 사용하는 것은 물론 새로운 기능을 추가할 수 있도록 구현이 되어있다. 무엇보다 GMES는 오픈 소스 프로그램만을 사용해서 만들어졌고, 자체의 저작권도 GPL을 따르고 있어서 추후 관련 연구자들이 자유롭게 코드를 이용할 수 있는 장점이 있다.

GMES는 특히, 플라즈몬 소자의 시뮬레이션을 목적으로 개발되어서 매우 정확하고 효율적인 분산매질 계산을 수행할 수 있다. 이는 드루드-임계점(Drude-critical point) 모델을 구현한 FDTD 알고리듬을 제공하기 때문이다. 이 구현은 부분선형재귀적합성곱(piecewise-linear recursive-convolution, PLRC) 방법과 보조미분방정식(auxiliary differential equation, ADE) 방법으로 모두 구현하여 각 구현의 장단점을 비교하였다. 두 구현 모두 시간스텝의 크기에 대해 2차 정확도를 보장함을 확인하여 기존의 비분산매질의 FDTD 알고리듬의 정확도를 유지하는 수준에서 분산매질의 정확한 계산이 가능함을 보였다.
